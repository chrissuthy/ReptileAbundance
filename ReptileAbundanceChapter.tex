\documentclass{book}
% test
\usepackage[explicit]{titlesec}
\usepackage{xcolor}
\usepackage{lipsum}% just to generate text
\usepackage{multirow}
%\colorlet{myrulecolor}{black}
\definecolor{myrulecolor}{RGB}{150,20,0}% define the color for the rules

\titleformat{\chapter}[display]
  {\normalfont\scshape\Huge}
  {\hspace*{-70pt}\thechapter.~#1}
  {-15pt}
  {\hspace*{-110pt}{\color{myrulecolor}\rule{\dimexpr\textwidth+80pt\relax}{3pt}}\Huge}
\titleformat{name=\chapter,numberless}[display]
  {\normalfont\scshape\Huge}
  {\hspace*{-70pt}#1}
  {-15pt}
  {\hspace*{-110pt}{\color{myrulecolor}\rule{\dimexpr\textwidth+80pt\relax}{3pt}}\Huge}
\titlespacing*{\chapter}{0pt}{0pt}{30pt}

\begin{document}
\chapter*{27. Estimating Abundance\\
          \noindent{\large{Chris Sutherland \& Andy Royle}}}

\section*{1. Introduction}
%intro: motivate CMR

Fundamental to much of applied ecology, including species management
and conservation, is the ability to reliably estimate population size,
or $abundance$. This is particulary true for reptiles given growing
evidence of declines globally (Gibbons et al. 2000, Reading et
al. 2010) and high levels of data deficiency (B{\"o}hm et a.l 2013). The 
major challenge arises because rarely are all the individuals in a 
population encountered during a survey, and that the
resulting $counts$, i.e. the number of individuals encountered, $C$,
represents only some fraction of the true abundance, $N$. Although the
distinction between counts and true abundance is an important one, the
two are intuitively related and, by repeatedly sampling and marking
individuals in a population, capture-recapture methods provide a
formalization of this relationship generating estimates of the true
population size.

In this chapter, we provide a non-technical overview of `closed
population capture-recapture' models, a class of well established
models that are widely applied in ecology (Borchers et al. 2002,
Williams et al. 2001), and regularly adopted for studies of reptiles
(Mazerolle et al. 2007), to estimate abundance from counts of marked
individuals by accounting for imperfect detection. We first describe
some classic closed population models for estimating abundance (Otis
et al., 1978), then consider some recent extensions that provide a
spatial context for the estimation of abundance, and therefore
density, $D$ (Spatial or spatially-explicit Capture-Recapture: Efford
2004, Royle et al. 2014), and finally provide an example of estimating
abundance and density of reptiles using an artificial cover object
survey of slow worms, \textit{Anguis fragilis}.

\subsection*{2. Closed Population Capture-Recapture}
\subsubsection*{2.1. Sampling a population}
%A typical CMR study

The primary objective in a closed population capture-recapture study is
to estimate the abundance of a population of interest. The population
of $N$ individuals is subjected to repeated sampling for a specified
number of occasions, say $K$ where, in the first sampling occasion,
all captured individuals are marked an released, and then at each
subsequent sampling occasion the detection of marked individuals is
recorded and unmarked individuals are marked. Identifying a focal
population, the spatial extent of the population is implicitly defined
and the method of capture depends on the species in question and
available resources. For reptiles, survey methods that allow
individuals to be captured and marked include, for example, visual
searches within a defined area (desert tortoise, horned lizards REFS),
cage and pit-fall traps (guam snakes is an example... XXX), and the use of
artificial cover objects () (see also Chapter X). Once captured,
individuals can be uniquely identified using either natural markings
that can be used to determine individual identity (), using tags or
colored markings () or physical marking such as toe clipping () (see
also Chapter X).

Such repeated sampling results in individual encounter histories that,
for each of the $i=1,\ldots,n$ individuals encountered, describes
whether or not individuals were detected in each of the $K$
occasions. For example, in a $K = 4$ occasion capture-recapture study,
an individual with an encounter $y_i = (0 1 0 1)$ was encountered
$\hbox{y}_i = 2$ times; first in occasion 2, and then again in
occasion 4, and was not encountered in occasions one or three. In
Table \ref{enchist} we provide an example of encounter history data
for a $K=4$ occasion capture-recapture study during which $n=8$
individuals were captured.

\begin{table}[h]
  \centering
  \caption{An example of an encounter history for a $K = 4$ occasion capture-recapture study during which $n=8$ individuals were detected.}
  \label{enchist}
 \begin{tabular}{lccccr}
 \hline
    &\multicolumn{5}{c}{Occasion} \\
  Individual & 1 & 2 & 3 & 4 & y\\
 \hline
  1   & 0 & 1 & 0 & 1 & 2 \\
  2   & 1 & 0 & 0 & 1 & 2 \\
  3   & 0 & 1 & 1 & 1 & 3 \\
  4   & 1 & 1 & 0 & 0 & 2 \\
  5   & 1 & 0 & 0 & 1 & 2 \\
  6   & 0 & 0 & 1 & 1 & 2 \\
  7   & 0 & 0 & 1 & 0 & 1 \\
  n=8 & 1 & 0 & 0 & 1 & 2 \\
 \hline
 \end{tabular}
\end{table}

Estimating abundance using encounter history data collected using the
general sampling scheme we have described above is basically the
process of estimating how many individuals were $missed$, i.e., how
many individuals have encounter history $\hbox{y}_i = 0$. The ability
to do so requires that the following basic assumptions are met:

\begin{enumerate}
\item the population is closed to demographic processes and to
  movement
\item individual marks can be identified unambiguously and are not lost
\item individuals are equally likely to be captured
\end{enumerate}

A `closed' population is one that experiences no additions or
subtractions for the duration of the study, and whose size is
therefore assumed to be fixed during sampling. Defining a sampling
period over which the assumption of closure can be satisfied means
that an individual detected at least once during the study was present
for the entire study, and therefore, failure to detect that individual
in any occasion was due to imperfect detection. This highlights the
importance of the second assumption -- that individuals are identified
unambiguously -- because misidentification would lead to erroneous
encounter histories that don't reflect the true process of
encountering individuals. The third assumption is less important as we
will see later but satisfying this assumption means that we can employ
the simplest formulation of a capture-recapture model, model $M_0$.

\subsubsection*{2.2. Estimating abundance using model $M_0$}

Under model $M_0$, the encounter probability for each individual,
$p_i$, is assumed to be the same for all individuals in the
population, i.e., $p_i = p$. This means that $y_{ik}$, the probability
of detecting individual $i$, during sampling occasion $k$ is the same
for every individual in the population ($i = 1,\ldots,N$), which we
can write formally as:
\[
y_{ik} \sim \hbox{Bernoulli}(p),
\]
which is to say individual encounter are Bernoulli random variables
with a constant $p$ (i.e., there are no individual or temporal
covariates that affect $p$). The basic idea of all closed population
capture-recapture methods  is that the pattern of
detections in the encounter histories of individuals observed at least
once provides information about individual detectability which, or
detection probability, $p$ which in
turn, can be used to estimate the number of individuals $not$
encountered.  The basic idea is that the observed number of
individuals $n$ is related to the total population size $N$ by the
expression:
\[
 E(n) = N\tilde{p}
\]
where $E()$ denotes statistical expectation and $\tilde{p}$ is the
probability that an individual is captured {\it at all} during the
study. In a study of $K$ survey occasions $\tilde{p}$ is directly
related to $p$ by the formula
\[
 \tilde{p} = 1-(1-p)^K
\]
This expression applies to model M0 but the
 expression relating $p$ to $\tilde{p}$ is different depending on
the specifici  capture-recapture being considered.

In general the parameter $p$ can be estimated from the observed
encounter histories and, in turn, this is used to estimate $\tilde{p}$
and then finally we estimate $N$ by $\hat{N} = n/\tilde{p}$.

XXXX working XXXX

Thus for example consider our individuals in the above
table where we see 8 individuals sampled over 4 periods producing a
total of 16 captures. Thus, on average, the capture probability for
these 8 individuals was 0.50. We note that this is actually {\it not}
an unbiased estimator of the probability of capture because we only
observe , in our data set, captured individuals and they necessarily
have at least one encounter event in their history. So, under the
binomial model, we have some all-zero encounter histories that we
don't observe and so our naive estimate of 0.50 is , in fact, biased
high. The true value should be much smaller. Part of the technical
ideas underling closed capture-recapture models is to account for this
bias in the observed sample and produce the correct unbiased estimate
of $p$. One way to do this is using the so-called conditional
likelihood and another approach is to use the 'full likelihood' which
actually has a few different variations.  In any case, without giving
these details (See Royle et al. 2014 sec. xxxx or Borchers et al...)
the conceptual motivation for capture-recapture is to estimate
 the probability of  capturing an individual {\it at all} that
exists
in this population from the observed encounter histories.
Call this  probability $p_{cap} = (1-p)^4$ where $p$ is the
per-occasion capture-probability, a fundamental parameter of the
model. We would estimate $p$ using the condietional likelihood or full
likelihood approaches and then just plug in $p$ into the expression
for $p_{cap}$ and we find that $p= 0.40$ (or whatever it is ) and
therefore
$p_{cap} =  .8$ (or whatever it is).

the conditional probability of encounter
 derives from a straightforward application of the law of total
probability. Conceptually, we partition $\Pr(y)$ according to
$\Pr(y) = \Pr(y|y>0)\Pr(y>0) + \Pr(y|y=0)\Pr(y=0)$. For any positive
value of $y$ the 2nd term is necessarily 0, and so we rearrange to
obtain
$\Pr(y|y>0) = \Pr(y)/\Pr(y>0)$ which, in the specific case where
$\Pr(y)$ is the

XXX end XXXXX

XXX I am unsure of the technical level expected is this chapter, so here we can chuck in some equations, or not. For example should we cycle through the various ways $M_0$ can be formulated (bern, Binom, Multinom etc...) XXX\\
XXXX probably not so important , the key point is the coin flipping
model just mentioned XXXX


Thus, in model $M_0$ there are two parameters of interest that are
estimated using individual encounter histories: $p$, the probability
of encountering an individual during a sampling occasion; and $n0$,
the number of individuals present in the population that went
unobserved (note that total population size is $N = n + n0$).


\subsubsection*{2.3. Variation in $p$: beyond $M_0$}

The assumption of equal capture probability is a rather restricting
one and there are many situations under which the capture
probabilities would be expected to vary. For these situations, model
$M_0$ is not appropriate although and Otis et al. (1978) described a
family of models that can be used to deal with most sources of
variation in individual encounter probabilities:

\begin{itemize}
\item[$M_0$] Capture probability is the same for all individuals
\item[$M_t$] Capture probability is the same for all individuals, but varies between sampling occasions (\textbf{t}ime)
\item[$M_b$] Capture probabilities vary depending on whether or not individuals have been captured previously (\textbf{b}ehavioral response)
\item[$M_h$] Capture probabilities vary among individuals (individual \textbf{h}eterogeneity)
\end{itemize}


\subsubsection*{2.4. Removal Sampling}


\subsection*{2.5 Individual covariate models and distance sampling}

Model $M_h$ is the standard closed population model when {\it
  unexplained} individual heterogeneity in capture-probability
exists. However an important related class of models are models in
which individual heterogeneity can be explained by explicit individual
covariates. These are often called ``individual covariate models''
but, in keeping with the classical nomenclature on closed population
models, Kery and Schaub (2012) referred to these models as ``model
$M_{x}$'', the $x$ representing some explicit covariate (of course
multiple covariates are allowed).  Classical examples of covariates
influencing detection probability are type of animal (juvenile/adult
or male/female), a continuous covariate such as body mass, or a
discrete covariate such as group or cluster size. For example, in
models of aerial survey data, it is natural to model the detection
probability of a group as a function of the observation-level
individual covariate, ``group size'' (Royle 2009, Langtimm et
al. 2011).

The basic encounter model for model $M_x$ is the same as our other
closed models, the Bernoulli encounter model:
\[
y_{i} \sim \mbox{Bernoulli}(p_{i}).
\]
To model the covariate, we use a logit model for encounter probability
of the form:
\begin{equation}
 \mbox{logit}(p_{i}) = \alpha_0 + \alpha_1 x_{i}
\end{equation}
where $x_i$ is the covariate value for individual $i$ and the
parameters $\alpha_0$ and $\alpha_1$ are the parameters to be
estimated.

Traditionally, estimation of $N$ in model $M_{x}$ is
achieved using methods based on ideas of unequal probability sampling
(i.e., Horvitz-Thompson estimation). This idea was developed
independently by Huggins (1989) and Alho (1990). The estimator of N is
given as a derived parameter:
\[
\hat{N} = \sum_{i=1}^{n} \frac{1}{\tilde{p}_{i}}
\]
where $\tilde{p}_{i}$ is the probability that individual $i$ appeared
in the sample.   This is related to the more fundamental parameters
$\alpha$ in the model for detection probability according to:
\[
\tilde{p}_{i}  = 1- (1-p_{i})^K
\]
where $p_{i}$ is a function of parameters $\alpha_{0}$ and $\alpha_{1}$.
In practice, parameters are
estimated from the conditional-likelihood of the observed encounter
histories.

An alternative formulation of model $M_x$ is the ``full likelihood''
which requires that we put a model on the individual covariate $x$
allowing for the sample not only of the encounter histories but also
of the covariate to be extrapolated to the population.
 For example, if we have a continuous trait measured on
each individual, then we might assume that $x$ has a normal distribution:
\[
x_{i} \sim \mbox{Normal}(\mu,\sigma^{2})
\]
If the covariate was group size then, naturally, some discrete
probability mass function would be needed. Inference for individual
covariate models from the standpoint of the  full likelihood is
discssed in Royle (2009), Kery and Schaub (2012), etc..

Individual covariate models are important in practice for the simple
reason that heterogeneity exists in almost every capture-recapture
study due to the spatial organization of traps and of individuals in
the population (see next section). Thus they were adopted historically
to account for spatial structure in capture-recapture
(Boullanger and McClellan XXXX, Karanth and Nichols XXX).
For this purpose an individual covariate is created which
describes {\it where} the individual is located in relation to the
trapping array.  This approach leads naturally to more recent spatial
capture-recapture models described in the next section.



\subsection*{3. Spatial Capture-Recapture}

The sampling scheme for a $spatial$ capture-recapture analysis is the same as described above, i.e., there is a population of $N$ individuals, but now we consider each individual having an activity center that has $X$ and $Y$ coordinates ($\textbf{s}_i=(s_{i,X},s_{i,Y})$). Now the goal is to estimate the number of individuals (or activity centers) within a region of interest which we refer to as a $state$-$space$, or ${\cal S}$, which is to say we wish to estimate density: $D = N/||{\cal S}||$, where $||{\cal S}||$ is the area of ${\cal S}$. We assume that these activity centers are distributed uniformly throughout across space:
\[
\textbf{s}_i \sim \hbox{Uniform}({\cal S}).
\]
As before, the population is subjected to sampling using some trapping devices (for convenience, we will refer to these as `traps'). However, we explicitly acknowledge both how many traps there are: $j=1,...,J$ traps, and the locations of each of the traps, which we will call these locations $x_j$. The acknowledgement of the spatial structure of the traps means observations can be spatially indexed so encounter histories describe $who$ ($i$), $when$ ($k$), and importantly, $where$ ($j$) individuals were located, i.e., $y_{i,j,k}$. Typically, these observations are assumed to be binomially distributed with sample size $K$ (the number of sampling occasions):
\[
y_{i,j} \sim \hbox{Binomial}(K,p_{i,j}),
\]
where $p_{i,j}$ is the probability of encountering individual $i$ in trap $j$, which depends on the distance between the trap location ($x_j$) and the individuals activity center ($s_i$) as follows:
\begin{equation}
p_{i,j} = p_0 \times e^{-\alpha_1 \hbox{d}(x_{j},s_{i})^{2}}.
\end{equation}
This is refered to as the `bivariate normal' encounter model where $\hbox{logit}(p_0) = \alpha_0$ is the baseline encounter probability, which is the probability of encountering an individual at its' activity center, $\alpha_1 = 1/(2\sigma^2)$ describes the distance over which detection declines, and $\hbox{d}(x_{j},s_{i})$ is the Euclidean distance between trap $j$ and the activity center of individual $i$. In a spatial capture-recapture analysis, the parameters to be estimated are $\alpha_0$ and $\alpha_1$.

XXX perhaps a comment about this model as a model of space use, and also, but perhaps not, spatial variation in density XXX

\subsection*{4. Software}

CPCR:\\
  MARK\\
  Rmark\\
  unmarked for fitting certain types of hierarchical capture-recapture models.\\
  BUGS/JAGS\\
SCR:\\
  Density superceded by secr\\
  oSCR\\

\subsection*{5. Slow worm example}

I'm doing this.....

\section{Summary}


MARK


We emphasize Bayesian analysis of capture-recapture models and we
accomplish this using a method related to classical ``data
augmentation'' from the statistics literature (e.g., tanner wong, 1987).  This is a general concept in
statistics but, in the context of capture-recapture models where $N$
is unknown, it has a consistent implementation across classes of
capture-recapture models and one that is really convenient from the
standpoint of doing MCMC
(royle etal, 2007; royle dorazio 2012). We use data augmentation
throughout this book and thus emphasize its conceptual and technical
origins and demonstrate applications to closed population models.  We
refer the reader to (Ch. 6 kery schaub 2011) for an
accessible and complementary development of Bayesian analysis of
ordinary, i.e., nonspatial closed population models.



\end{document}



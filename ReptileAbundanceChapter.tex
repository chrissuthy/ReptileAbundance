\documentclass{book}
\usepackage[explicit]{titlesec}
\usepackage{xcolor}
\usepackage{lipsum}% just to generate text

%\colorlet{myrulecolor}{black}
\definecolor{myrulecolor}{RGB}{150,20,0}% define the color for the rules

\titleformat{\chapter}[display]
  {\normalfont\scshape\Huge}
  {\hspace*{-70pt}\thechapter.~#1}
  {-15pt}
  {\hspace*{-110pt}{\color{myrulecolor}\rule{\dimexpr\textwidth+80pt\relax}{3pt}}\Huge}
\titleformat{name=\chapter,numberless}[display]
  {\normalfont\scshape\Huge}
  {\hspace*{-70pt}#1}
  {-15pt}
  {\hspace*{-110pt}{\color{myrulecolor}\rule{\dimexpr\textwidth+80pt\relax}{3pt}}\Huge}
\titlespacing*{\chapter}{0pt}{0pt}{30pt}

\begin{document}
\chapter*{27. Estimating Abundance\\
          \noindent{\large{Chris Sutherland \& Andy Royle}}}
\begin{itemize}
\item{repeated sampling of individuals}
\item{counts versus \emph{true} abundance}
\item{a heuristic estimator}
\item{flavours of p
  \item{time}
  \item{behaviour}
  \ldots
  \item{individual covariates}
}
\item{distance to trap as an individual covariate}
\item{enter SCR}
\item{the promise of SCR}
\item{An example - lizards??}


\end{itemize}
Importance of differentiating between counts and abundance; pitfalls of misinterpreting such data; detection probabilities. Capture-mark-recapture; removal sampling.


\section{Introduction}

Inference about population size or abundance is a fundamental
objective in applied ecology and management of species.
Capture-recapture methods have been widely applied to that end for
many decades (Otis et al. 1978; Seber 1982; Williams et al. 2002).
Capture-recapture methods can be characterized by whether they are
``open'' or ``closed'' population models.  Open populations are those
which experience net recruitment or mortality over the period of
sampling so that individuals are added or removed from the population
and $N$ may not be constant. Conversely, a closed population is one
with no additions or removals to the population and in which $N$ does
not change during the study.  Two forms of closure are often
discussed: demographic closure, meaning that no births or deaths
occur, and geographic closure, which states that no individuals move
onto or off of the sampled area during the study.  Classical closed
population models assume both of these variations of closure although
newly developed spatial capture-recapture models relax the assumption
of geographic closure.  Although few populations are actually closed
except during very short time intervals, closed population CR models
serve as the basis for the development of broad classes of models
including models for open populations and recent spatial
capture-recapture models (Royle et al. 2014).

Relevance of capture-recapture to herptile populations..... marking,
capturing methods... removal...

In this chapter we review conventional types of closed population
models for estimating population size including the basic models M0 Mb
Mt Mh and related models described in the synthesis by Otis et
al. (1978).  We discuss special cases such as distance sampling (a
type of individual covariate model) used widely for sampling tortoise
populations. Removal methods... and finally we provide a brief
overview of spatial capture recapture (SCR or 'secr') models which
have been developed over the last decade (Efford 2004). 


capture-recapture model, colloquially referred to as ``model $M_0$''
\citep{otis_etal:1978}, in which encounter probability is strictly
constant in all respects (across individuals, and replicates).  This
allows us to highlight the basic structure of closed population models
as binomial GLMs.  We then consider some important extensions of
ordinary closed population models that accommodate various types of
``individual effects'' --- either in the form of explicit, observed
covariates (sex, age, body mass) or unstructured ``heterogeneity'' in
the form of an individual random effect, which represent 
unobserved or unmeasured covariates.  A special type of individual
covariate models is distance sampling, which could be thought of as
the most primitive spatial capture-recapture model.  All of these
different types of closed population models are closely related to
binomial (or logistic) regression-type models. In fact, when $N$ is
known, they are precisely logistic regression models.









\section{Software and analysis}

MARK

unmarked for fitting certain types of hierarchical capture-recapture
models. 

We emphasize Bayesian analysis of capture-recapture models and we
accomplish this using a method related to classical ``data
augmentation'' from the statistics literature
\citep[e.g.,][]{tanner_wong:1987}.  This is a general concept in
statistics but, in the context of capture-recapture models where $N$
is unknown, it has a consistent implementation across classes of
capture-recapture models and one that is really convenient from the
standpoint of doing MCMC
\citep{royle_etal:2007,royle_dorazio:2012}. We use data augmentation
throughout this book and thus emphasize its conceptual and technical
origins and demonstrate applications to closed population models.  We
refer the reader to \citet[][ch. 6]{kery_schaub:2011} for an
accessible and complementary development of Bayesian analysis of
ordinary, i.e., nonspatial closed population models.









\end{document}


